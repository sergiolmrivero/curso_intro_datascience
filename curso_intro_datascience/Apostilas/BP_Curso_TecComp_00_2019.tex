\documentclass[12pt,a4paper,oneside]{erdc}
%\documentclass{erdc}
\usepackage[T1]{fontenc}		% Selecao de codigos de fonte.
\usepackage[utf8]{inputenc}		% Codificacao do documento (conversão automática dos acentos)
%\usepackage{lastpage}			% Usado pela Ficha catalográfica
\usepackage{indentfirst}		% Indenta o primeiro parágrafo de cada seção.
\usepackage{color}				% Controle das cores
\usepackage{graphicx}			% Inclusão de gráficos
%\usepackage{microtype} 			% para melhorias de justificação
\usepackage[brazil]{babel}
%\usepackage[brazilian,hyperpageref]{backref}	 % Paginas com as citações na bibl
%\usepackage[alf]{abntex2cite}	% Citações padrão ABNT
\usepackage{hyperref}
\usepackage{natbib}

\usepackage{lipsum}


\begin{document}

\frontmatter

\laboratory{PPGE-UFPA}

\reportnum{BP/EcoS - Curso-DataScience-00-2019}

\program{Construção de Modelos e Indicadores Econômicos}

\title{Introdução ao Tratamento e Análise de Dados em R}

%\subtitle{Assessments and Report from Socioeconomic and Demographic Data in \\
%	      Small Cities in Amazon Delta}

\subtitle{ou Data Science para todos!}

\date{\today}

\author{S.~Rivero \and H.~Farias}

\affiliation{Programa de Pós-Graduação em Economia\\
  Instituto de Ciências Sociais Aplicadas\\
  Universidade Federal do Pará\\
  Rua Augusto Correia, 1\\
  Belém, Pará - 66.075-200}

\author{Equipe UFPA }


\affiliation{Faculdade de Economia \\
	Instituto de Ciências Sociais Aplicadas\\
	Universidade Federal do Pará\\
	Rua Augusto Correia, 1\\
	Belém, Pará - 66.075-200}


\coverart[width=\linewidth]{../figs/Capa}

\reporttype{Produto: Cursos}

\distribution{Propriedade BANPARÁ e PPGE-UFPA \\ (Distribuição Restrita)}

% \distribution{Distribution authorized to U.S. Government Agencies
% only; Test and Evaluation; November 2005.  Other requests should be
% referred to U.S. Army Engineer Research and Development Center}

%\additionalinfo{Supersedes ERDC/CREL AF-01-23}

\begin{abstract}
  \lipsum[12-13]
\end{abstract}

\disclaimer{
	
	This document is an output from the  Banpará Project 
	
	PPGE-UFPA report
	
	Distribution Restrictions
	
	\copyright 2019, All rights reserved
	
	\pagebreak
	
	}

\preparedfor{Banpará} 

\contractnum{FADESP-NO-CONTRATO}

\monitoredby{Banpará}

%\preparedfor{}




\maketitle

\tableofcontents


%\chapter*{Resumo Executivo}


\mainmatter


\chapter{Introdução }

Para que serve o R?

Como se pode utilizar o conjunto de pacotes R para Análise de Dados e BI?

Uma configuração adequada para usar ferramentas de BI 

		

\chapter{Aula 1 - Instalando e Configurando o R e RStudio}

Nesta aula os alunos aprenderão a baixar o R e RStudio bem como aprenderão a utilizar bibliotecas em R

\section{Chegando a Instalação Existente e os Requisitos}

\section{Instalando o R e RStudio}

\section{Bibliotecas no R}

\subsection{O conceito de Biblioteca e para que serve}

\subsection{Como sei que bibliotecas eu preciso?}

\subsection{Baixando as Bibliotecas}

\subsection{Resolvendo problemas de compilação}

\subsection{Utilizando as bibliotecas no seu programa R}


\chapter{Aula 2 - Acessando e Utilizando Bases de Dados}

Apresentar o conceito de Dataframe, os tipos de dados utilizados no R e os principais comandos  

\section{O ciclo de tratamento e análise de dados}

\section{Tipos de Dados em R}

\section{Dataframes}

\section{Acessando Arquivos no computador}


\section{Acessando Bases de dados via \textit{APIs}}

\section{Trabalhando com bases de dados muito grandes}


\chapter{Aula 3 - Limpando e organizando seus dados}

\section{O que é uma boa base de dados e que tipos de bases existem?}

\section{dplyr}

\section{tidyr}

\section{tidyverse}

\chapter{Aula 4 - Apresentando Resultados}

\section{Rmarkdown - Preparando o relatório enquanto você analisa os dados}

\section{Gerando dados sintetizados - Utilizando dataframes}

\section{Gerando Tabelas}

\subsection{xtable}

\subsection{stargazer}

\chapter{Aula 5 - Gerando estatísticas dos Dados}

\section{Estatísticas Descritivas}

\section{Correlação}

\section{Apresentando os Resultados em Gráficos}

\subsection{Gráficos Simples}

\subsection{Bibliotecas de Gráficos}

\chapter{Aula 6 - Apresentando Resultados}

\section{ggplot}

\section{shiny}

\chapter{Aula 7 - Utilizando modelos}

\section{Regressão}

\subsection{Executando a Regressão}

\subsection{Apresentando os resultados da Regressão}

\section{Gerando Dashboards}

\chapter{Aula 8 - Encerramento do Curso}


\chapter{Onde aprender mais?}


  
  
  

%\bibliographystyle{unsrt}
\bibliographystyle{alpha}
\bibliography{../bib/bibliografia}

%\appendix

%\chapter{Tabelas demográficas}

%\lipsum[12-16]

%\begin{equation}
%  \label{eq:C}
%  \frac{\partial H}{\partial x} = -X
%\end{equation}

%\chapter{PeCiDAm}

%\lipsum[16-22]




\end{document}
