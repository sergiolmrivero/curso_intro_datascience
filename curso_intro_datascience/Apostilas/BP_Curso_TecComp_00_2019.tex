\documentclass[12pt,a4paper,oneside]{erdc}\usepackage[]{graphicx}\usepackage[]{color}
%% maxwidth is the original width if it is less than linewidth
%% otherwise use linewidth (to make sure the graphics do not exceed the margin)
\makeatletter
\def\maxwidth{ %
  \ifdim\Gin@nat@width>\linewidth
    \linewidth
  \else
    \Gin@nat@width
  \fi
}
\makeatother

\definecolor{fgcolor}{rgb}{0.345, 0.345, 0.345}
\newcommand{\hlnum}[1]{\textcolor[rgb]{0.686,0.059,0.569}{#1}}%
\newcommand{\hlstr}[1]{\textcolor[rgb]{0.192,0.494,0.8}{#1}}%
\newcommand{\hlcom}[1]{\textcolor[rgb]{0.678,0.584,0.686}{\textit{#1}}}%
\newcommand{\hlopt}[1]{\textcolor[rgb]{0,0,0}{#1}}%
\newcommand{\hlstd}[1]{\textcolor[rgb]{0.345,0.345,0.345}{#1}}%
\newcommand{\hlkwa}[1]{\textcolor[rgb]{0.161,0.373,0.58}{\textbf{#1}}}%
\newcommand{\hlkwb}[1]{\textcolor[rgb]{0.69,0.353,0.396}{#1}}%
\newcommand{\hlkwc}[1]{\textcolor[rgb]{0.333,0.667,0.333}{#1}}%
\newcommand{\hlkwd}[1]{\textcolor[rgb]{0.737,0.353,0.396}{\textbf{#1}}}%
\let\hlipl\hlkwb

\usepackage{framed}
\makeatletter
\newenvironment{kframe}{%
 \def\at@end@of@kframe{}%
 \ifinner\ifhmode%
  \def\at@end@of@kframe{\end{minipage}}%
  \begin{minipage}{\columnwidth}%
 \fi\fi%
 \def\FrameCommand##1{\hskip\@totalleftmargin \hskip-\fboxsep
 \colorbox{shadecolor}{##1}\hskip-\fboxsep
     % There is no \\@totalrightmargin, so:
     \hskip-\linewidth \hskip-\@totalleftmargin \hskip\columnwidth}%
 \MakeFramed {\advance\hsize-\width
   \@totalleftmargin\z@ \linewidth\hsize
   \@setminipage}}%
 {\par\unskip\endMakeFramed%
 \at@end@of@kframe}
\makeatother

\definecolor{shadecolor}{rgb}{.97, .97, .97}
\definecolor{messagecolor}{rgb}{0, 0, 0}
\definecolor{warningcolor}{rgb}{1, 0, 1}
\definecolor{errorcolor}{rgb}{1, 0, 0}
\newenvironment{knitrout}{}{} % an empty environment to be redefined in TeX

\usepackage{alltt}
%\documentclass{erdc}
\usepackage[T1]{fontenc}		% Selecao de codigos de fonte.
\usepackage[utf8]{inputenc}		% Codificacao do documento (conversão automática dos acentos)
%\usepackage{lastpage}			% Usado pela Ficha catalográfica
\usepackage{indentfirst}		% Indenta o primeiro parágrafo de cada seção.
\usepackage{color}				% Controle das cores
\usepackage{graphicx}			% Inclusão de gráficos
%\usepackage{microtype} 			% para melhorias de justificação
\usepackage[brazil]{babel}
%\usepackage[brazilian,hyperpageref]{backref}	 % Paginas com as citações na bibl
%\usepackage[alf]{abntex2cite}	% Citações padrão ABNT
\usepackage{hyperref}
\usepackage{natbib}

\usepackage{lipsum}
\IfFileExists{upquote.sty}{\usepackage{upquote}}{}
\begin{document}

\frontmatter

\laboratory{PPGE-UFPA}

\reportnum{BP/EcoS - Curso-DataScience-00-2019}

\program{Construção de Modelos e Indicadores Econômicos}

\title{Introdução ao Tratamento e Análise de Dados em R}

%\subtitle{Assessments and Report from Socioeconomic and Demographic Data in \\
%	      Small Cities in Amazon Delta}

\subtitle{ou Data Science para todos!}

\date{\today}

\author{S.~Rivero \and H.~Farias}

\affiliation{Programa de Pós-Graduação em Economia\\
  Instituto de Ciências Sociais Aplicadas\\
  Universidade Federal do Pará\\
  Rua Augusto Correia, 1\\
  Belém, Pará - 66.075-200}

\author{Equipe UFPA }


\affiliation{Faculdade de Economia \\
	Instituto de Ciências Sociais Aplicadas\\
	Universidade Federal do Pará\\
	Rua Augusto Correia, 1\\
	Belém, Pará - 66.075-200}


\coverart[width=\linewidth]{../figs/Capa}

\reporttype{Produto: Cursos}

\distribution{Propriedade BANPARÁ e PPGE-UFPA \\ (Distribuição Restrita)}

% \distribution{Distribution authorized to U.S. Government Agencies
% only; Test and Evaluation; November 2005.  Other requests should be
% referred to U.S. Army Engineer Research and Development Center}

%\additionalinfo{Supersedes ERDC/CREL AF-01-23}

\begin{abstract}
  \lipsum[12-13]
\end{abstract}

\disclaimer{
	
	This document is an output from the  Banpará Project 
	
	PPGE-UFPA report
	
	Distribution Restrictions
	
	\copyright 2019, All rights reserved
	
	\pagebreak
	
	}

\preparedfor{Banpará} 

\contractnum{FADESP-NO-CONTRATO}

\monitoredby{Banpará}

%\preparedfor{}




\maketitle

\tableofcontents


%\chapter*{Resumo Executivo}


\mainmatter


\chapter{Introdução }

Para que serve o R?

Como se pode utilizar o conjunto de pacotes R para Análise de Dados e BI?

Uma configuração adequada para usar ferramentas de BI 

\chapter{Aula 1 - Instalando e Configurando o R e RStudio}

Nesta aula os alunos aprenderão a baixar o R e RStudio bem como aprenderão a utilizar bibliotecas em R

\section{Checando a Instalação Existente e os Requisitos}l

\url{https://stackoverflow.com/questions/11103189/how-to-find-out-which-package-version-is-loaded-in-r}

\url{https://www.r-bloggers.com/list-of-user-installed-r-packages-and-their-versions/}

\url{https://community.rstudio.com/t/reinstalling-packages-on-new-version-of-r/7670}

\section{Instalando o R e RStudio}

\subsection{Instalando o R}

\url{https://a-little-book-of-r-for-bioinformatics.readthedocs.io/en/latest/src/installr.html}

\url{https://cran.r-project.org/doc/manuals/R-admin.html}


\subsection{Instalando o R Studio}

\url{https://www.rstudio.com/products/rstudio/}

\url{http://web.cs.ucla.edu/~gulzar/rstudio/}

\url{https://www.rstudio.com/products/rstudio/download/}

\url{http://rprogramming.net/download-and-install-rstudio/}

\section{Pacotes no R}

\subsection{O conceito de pacote e para que serve}

\url{https://www.datacamp.com/community/tutorials/r-packages-guide}

\url{https://www.rstudio.com/products/rpackages/}

\url{http://r-pkgs.had.co.nz/}

\subsection{Como sei que pacotes eu preciso?}

\url{https://blog.revolutionanalytics.com/2017/01/cran-10000.html}

\url{https://cran.r-project.org/web/packages/available_packages_by_name.html}

\url{https://cran.r-project.org/web/packages/}

\subsection{Baixando os pacotes}

\url{https://www.r-bloggers.com/installing-r-packages/}

\url{https://www.r-bloggers.com/how-to-install-and-include-an-r-package/}

\url{http://kbroman.org/pkg_primer/pages/build.html}


\subsection{Resolvendo problemas de compilação}

\url{https://stackoverflow.com/questions/23135703/package-install-error-compilation-failed}

\url{https://support.rstudio.com/hc/en-us/community/posts/200522573-Can-t-install-packages}

\url{http://mazamascience.com/WorkingWithData/?p=1185}


\subsection{Utilizando os pacotes no seu programa R}

\url{https://www.statmethods.net/interface/packages.html}

\url{https://www.dummies.com/programming/r/how-to-install-load-and-unload-packages-in-r/}


		
\chapter{Aula 2 - Acessando e Utilizando Bases de Dados}

Apresentar o conceito de Dataframe, os tipos de dados utilizados no R e os principais comandos  

\section{O ciclo de tratamento e análise de dados}

\section{Tipos de Dados em R}

\url{https://www.statmethods.net/input/datatypes.html}

\url{https://swcarpentry.github.io/r-novice-inflammation/13-supp-data-structures/}

\url{https://www.tutorialspoint.com/r/r_data_types.htm}

\url{http://www.r-tutor.com/r-introduction/basic-data-types}

\url{https://www.cyclismo.org/tutorial/R/types.html}

\url{https://stat.ethz.ch/R-manual/R-devel/library/base/html/typeof.html}

\section{Dataframes}

\url{https://www.tutorialspoint.com/r/r_data_frames.htm}

\url{https://www.datamentor.io/r-programming/data-frame/}

\url{http://www.r-tutor.com/r-introduction/data-frame}

\url{https://stat.ethz.ch/R-manual/R-devel/library/base/html/data.frame.html}

\url{https://www.tutorialgateway.org/data-frame-in-r/}

\url{https://datacarpentry.org/R-ecology-lesson/02-starting-with-data.html}

\url{https://www.statmethods.net/input/importingdata.html}

\section{Acessando Arquivos no computador}

\url{https://www.datacamp.com/community/tutorials/r-data-import-tutorial?utm_source=adwords_ppc&utm_campaignid=1455363063&utm_adgroupid=65083631748&utm_device=c&utm_keyword=&utm_matchtype=b&utm_network=g&utm_adpostion=1t1&utm_creative=332602034364&utm_targetid=dsa-473406573035&utm_loc_interest_ms=&utm_loc_physical_ms=1001610&gclid=Cj0KCQiA5NPjBRDDARIsAM9X1GLkgYWekNMkjHQnsTnRAzV7_gVEiwAqyW9CPisvAqFv2mNXzwarSlIaAgdZEALw_wcB}

\url{http://rprogramming.net/read-csv-in-r/}

\url{https://www.rdocumentation.org/packages/gdata/versions/2.18.0/topics/read.xls}

\url{https://stat.ethz.ch/R-manual/R-devel/library/utils/html/read.fwf.html}

\url{https://riptutorial.com/r/example/31447/importing-fixed-width-files}


\section{Acessando Bases de dados via \textit{APIs}}

\url{https://www.r-bloggers.com/accessing-apis-from-r-and-a-little-r-programming/}

\url{https://cran.r-project.org/web/packages/httr/vignettes/api-packages.html}

\url{https://zapier.com/learn/apis/}

\url{https://www.earthdatascience.org/courses/earth-analytics/get-data-using-apis/API-data-access-r/}



\section{Trabalhando com bases de dados muito grandes}

\url{http://dept.stat.lsa.umich.edu/~jerrick/courses/stat701/notes/sql.html}

\url{https://datacarpentry.org/R-ecology-lesson/05-r-and-databases.html}

\url{https://db.rstudio.com/}

\url{http://www.columbia.edu/~sjm2186/EPIC_R/EPIC_R_BigData.pdf}

\url{https://www.rstudio.com/resources/webinars/working-with-big-data-in-r/}

\url{https://rpubs.com/msundar/large_data_analysis}



\chapter{Aula 3 - Limpando e organizando seus dados}

	\section{O que é uma boa base de dados e que tipos de bases existem?}
	
	\section{dplyr}
	
	\section{tidyr}
	
	\section{tidyverse}




\chapter{Aula 4 - Apresentando Resultados}

	\section{Rmarkdown - Preparando o relatório enquanto você analisa os dados}
	
	\section{Gerando dados sintetizados - Utilizando dataframes}
	
	\section{Gerando Tabelas}
	
		\subsection{xtable}
		
		\subsection{stargazer}




\chapter{Aula 5 - Gerando estatísticas dos Dados}

	\section{Estatísticas Descritivas}
	
	\section{Correlação}
	
	\section{Apresentando os Resultados em Gráficos}
	
		\subsection{Gráficos Simples}
		
		\subsection{Bibliotecas de Gráficos}


\chapter{Aula 6 - Apresentando Resultados}

	\section{ggplot}

	\section{shiny}



\chapter{Aula 7 - Utilizando modelos}

	\section{Regressão}
	
		\subsection{Executando a Regressão}
		
		\subsection{Apresentando os resultados da Regressão}
	
	\section{Gerando Dashboards}




\chapter{Aula 8 - Encerramento do Curso}




\chapter{Onde aprender mais?}



%\bibliographystyle{unsrt}
\bibliographystyle{alpha}
\bibliography{../bib/bibliografia}

%\appendix

%\chapter{Tabelas demográficas}

%\lipsum[12-16]

%\begin{equation}
%  \label{eq:C}
%  \frac{\partial H}{\partial x} = -X
%\end{equation}

%\chapter{PeCiDAm}

%\lipsum[16-22]




\end{document}
